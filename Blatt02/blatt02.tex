\documentclass[a4paper]{article}
\usepackage[utf8]{inputenc}
\usepackage{amssymb}
\usepackage[ngerman]{babel}
\usepackage{hyperref}
\usepackage{enumitem}
\usepackage{listings}
\usepackage{esvect}
\usepackage{float}
\usepackage{graphicx}
\usepackage[table]{xcolor}% http://ctan.org/pkg/xcolor
\usepackage{todonotes}
\usepackage{pgfplots}
\usepackage{verbatim}
\usepackage{multirow}
\usepackage{booktabs}
\pgfplotsset{compat=1.10}
\usepgfplotslibrary{fillbetween}
\usetikzlibrary{patterns}
\usepackage{mathtools}
\usepackage{centernot}

\hypersetup{
     colorlinks   = true,
     citecolor    = gray
}


\title{Grundlagen der Rechnerarchitektur Blatt 2}
\author{Marco Deuscher \and Carolin Schindler}
\date{04. November 2019}

\begin{document}

\maketitle

\section{Aufgabe: Additive Zahlensysteme}

\section{Aufgabe: Polyadsiche Zahlensysteme}


\section{Aufgabe: Wir wollen dann bitte Zahlen...}
\paragraph{(a)}
$BAD55$\\
Hierbei ist $D$ das höchstwertige Symbol, damit muss $b\geq 14$ sein. Es gibt somit drei Interpretationsmöglichkeiten der Zahl ($b=14$,$b=15$,$b=16$).

\paragraph{(b)}
$DEADC0DE$\\
Hierbei ist $E$ das höchstwertige Symbol, damit muss $b\geq 15$ sein. Es gibt somit zwei Interpreationsmöglichkeiten der Zahl ($b=15$, $b=16$).

\paragraph{(c)}
$C0CAC01A$\\
Hierbei ist $C$ das höchstwertige Symbol, damit muss $b\geq	13$ sein. Es gibt somit vier Interpreationsmöglichkeiten der Zahl ($b=13$, $b=14$, $b=15$, $b=16$).

\section{Aufgabe: Bitwertigkeit}
In dieser Aufgabe gilt also $*=0$, $\#=1$, $\sim=2$ und $\$=3$

\paragraph{(a)}

\begin{enumerate}
	\item $ \$ \sim *\# = \$\cdot 4^3 + \sim\cdot4^2 + *\cdot 4^1 + \#\cdot4^0 = 4\cdot 64 + 2\cdot 16 + 0 + 1 = 289$ 
	\item $ \#\sim\sim\$ = \#\cdot4^3 + \sim\cdot4^2 + \sim\cdot4^1 + \$\cdot4^0 = 64 + 32 + 4 + 3 = 193$
	\item $ \#\$\$\sim * \sim = \#\cdot4^5 + \$\cdot 4^4 + \$\cdot 4^3 + \sim\cdot4^2 + *\cdot4^1 + \sim\cdot4^0 =1024 + 3\cdot256 + 3\cdot 64 + 2\cdot 16 + 0 + 2 = 2018  $	
\end{enumerate}

\paragraph{(b)}

\begin{enumerate}
	\item 
\end{enumerate}


      
\end{document}
