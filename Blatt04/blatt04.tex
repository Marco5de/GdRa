\documentclass[a4paper]{article}
\usepackage[utf8]{inputenc}
\usepackage{amssymb}
\usepackage[ngerman]{babel}
\usepackage{hyperref}
\usepackage{enumitem}
\usepackage{listings}
\usepackage{amsmath}
\usepackage{esvect}
\usepackage{float}
\usepackage{graphicx}
\usepackage[table]{xcolor}% http://ctan.org/pkg/xcolor
\usepackage{todonotes}
\usepackage{pgfplots}
\usepackage{verbatim}
\usepackage{multirow}
\usepackage{booktabs}
\pgfplotsset{compat=1.10}
\usepgfplotslibrary{fillbetween}
\usetikzlibrary{patterns}
\usepackage{mathtools}
\usepackage{centernot}
\usepackage{mathabx}

\newcommand{\uproman}[1]{\uppercase\expandafter{\romannumeral#1}}
\newcommand\mathbfont{\usefont{U}{mathb}{m}{n}}

\hypersetup{
     colorlinks   = true,
     citecolor    = gray
}


\title{Grundlagen der Rechnerarchitektur Blatt 4}
\author{Marco Deuscher \and Carolin Schindler}
\date{18. November 2019}

\begin{document}

\maketitle
\section{Aufgabe: Negativ, Positiv: So viele Möglichkeiten}
\paragraph{(a)} $11000101010_2$
\begin{align*}
\text{vorzeichenbehaftet: } &-554_{10} \\
	&\text{negative Zahl mit Betrag: } 1000101010_{2}\rightarrow (2 + 2^3 + 2^5 + 2^9)_{10} = 554_{10} \\
\text{b-1-Komplement: } &-469_{10} \\
	&\text{negative Zahl mit Betrag: } 00111010101_{2}\rightarrow (1 + 2^2 + 2^4 + 2^6 + 2^7 + 2^8)_{10} = 469_{10} \\
\text{b-Komplement: } &-470_{10} \\
	&\text{\glqq b-1-Komplement -1\grqq}: -469_{10} -1_{10}
\end{align*}

\paragraph{(b)} $01111010_2$
\begin{align*}
\text{vorzeichenbehaftet: } &122_{10} \\
	&\text{positive Zahl mit Betrag: } 1111010_{2}\rightarrow (2 + 2^3 + 2^4 + 2^5 + 2^6)_{10} = 122_{10} \\
\text{b-1-Komplement: } &122_{10} \\
	&\text{positive Zahl mit Betrag: } 01111010_{2}\rightarrow (2 + 2^3 + 2^4 + 2^5 + 2^6)_{10} = 122_{10} \\
\text{b-Komplement: } &121_{10} \\
	&\text{\glqq b-1-Komplement -1\grqq}: 122_{10} -1_{10}
\end{align*}

\paragraph{(c)} $1111111_2$
\begin{align*}
\text{vorzeichenbehaftet: } &-63_{10} \\
	&\text{negative Zahl mit Betrag: } 111111_{2}\rightarrow (1 + 2 + +2^2 + 2^3 + 2^4 + 2^5)_{10} = 63_{10} \\
\text{b-1-Komplement: } &-0_{10} \\
&\text{negative Zahl mit Betrag: } 0000000_{2}\rightarrow 0_{10} \\
\text{b-Komplement: } &-1_{10} \\
	&\text{\glqq b-1-Komplement -1\grqq}: -469_{10} -1_{10}
\end{align*}


\section{Aufgabe: Multiplikation und Division}
\paragraph{(a)}
\begin{align*}
XXX
\end{align*}
\paragraph{(b)}
\begin{align*}
XXX
\end{align*}


\section{Keine Brüche, nur Kommas}
\paragraph{(a)}
$1,453125_{10}\rightarrow 000001011101$ (ohne Abschneiden)
\begin{align*}
	0,453125 * 2 &= 0,90625 \\
	0,90625 * 2 &= 1,8125 \\
	0,8125 * 2 &= 1,625 \\
	0,625 * 2 &= 1,25 \\
	0,25 * 2 &= 0,5 \\
	0,5 * 2 &= 1 \\
\end{align*}
\paragraph{(b)}
$0,\overline{3}_{10}\rightarrow 000000010101_{2}$ (mit Abschneiden)
\begin{align*}
	\frac{1}{3}\cdot 2 &= \frac{2}{3} \\
	\frac{2}{3}\cdot 2 &= \frac{4}{3} \\
	\frac{1}{3}\cdot 2 &= \frac{2}{3} \\
	\frac{2}{3}\cdot 2 &= \frac{4}{3} \\
	\text{\ldots}
\end{align*}
Es gibt (abgesehen von der Einführung eines Periodenzeichens: $0,\overline{3}_{10}\rightarrow 000000\overline{01}_{2}$) keine Möglichkeit die Zahl als 12 Bit Festkommazahl darzustellen.



\section{Multiplizieren und Dividieren, aber schnell}
\paragraph{(a)}
$1001010100_{2}$ (entspricht $\ll 1_{10}$)
\paragraph{(b)}
$010100_{2}$ (entspricht $\ll 2_{10}$)
\paragraph{(c)}
$000000000001_{2}$ (entspricht $\gg 9_{10}$)
\paragraph{(d)}
XXX


\section{Binär und doch Dezimal}
\paragraph{(a)}
\begin{align*}
XXX
\end{align*}
\paragraph{(b)}
\begin{align*}
XXX
\end{align*}
\paragraph{(c)}
\begin{align*}
XXX
\end{align*}
\paragraph{(d)}
\begin{align*}
XXX
\end{align*}


\section{Was passiert hier?}
\paragraph{(a)}
\begin{align*}
XXX
\end{align*}
\paragraph{(b)}
\begin{align*}
XXX
\end{align*}

\section{Knobelaufgabe}
Es gibt Zahlen, die im Dezimalsystem weder irrational noch periodisch sind und im Dualsystem nicht durch eine endliche Anzahl an Stellend darstellbar sind. Ein Beispiel hierfür ist die Zahl $0,1_{10} \rightarrow 0,0\overline{0011}_{2}$:
\begin{align*}
0,1 \cdot 2 &= 0,2 \\
0,2 \cdot 2 &= 0,4 \\
0,4 \cdot 2 &= 0,8 \\
0,8 \cdot 2 &= 1,6 \\
0,6 \cdot 2 &= 1,2 \\
0,2 \cdot 2 &= 0,4 \\
\text{\ldots}
\end{align*}


\end{document}
