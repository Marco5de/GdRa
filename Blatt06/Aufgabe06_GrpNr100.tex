\documentclass[a4paper]{article}
\usepackage[utf8]{inputenc}
\usepackage{amssymb}
\usepackage[ngerman]{babel}
\usepackage{hyperref}
\usepackage{enumitem}
\usepackage{listings}
\usepackage{amsmath}
\usepackage{esvect}
\usepackage{float}
\usepackage{graphicx}
\usepackage[table]{xcolor}% http://ctan.org/pkg/xcolor
\usepackage{todonotes}
\usepackage{pgfplots}
\usepackage{verbatim}
\usepackage{multirow}
\usepackage{booktabs}
\pgfplotsset{compat=1.10}
\usepgfplotslibrary{fillbetween}
\usetikzlibrary{patterns}
\usepackage{mathtools}
\usepackage{centernot}
\usepackage{mathabx}

\newcommand{\uproman}[1]{\uppercase\expandafter{\romannumeral#1}}
\newcommand\mathbfont{\usefont{U}{mathb}{m}{n}}

\hypersetup{
     colorlinks   = true,
     citecolor    = gray
}


\title{Grundlagen der Rechnerarchitektur Blatt 5}
\author{Marco Deuscher \and Carolin Schindler}
\date{05. Dezember 2019}

\begin{document}

\maketitle
\section{Eine Schaltung für den Weihnachtsbaum}

\paragraph{(a) Wahrheitstafel}\mbox{}\\
$x_i$ kodieren Tag und $s_i$ ist Segment i
\begin{align*}
	\begin{array}{r|cccc|ccccccc}
		\text{Tag}& x_3&x_2&x_1&x_0 &s_1&s_2&s_3&s_4&s_5&s_6&s_7\\
		\hline
		\text{So } 01 &0&0&0&0 &1&1&1&1&1&1&1\\
		\text{Mo } 02 &0&0&0&1 &0&1&0&1&0&1&0\\
		\text{Di } 03 &0&0&1&0 &0&1&0&1&0&1&0\\
		\text{Mi } 04 &0&0&1&1 &1&1&0&1&0&1&0\\
		\text{Do } 05 &0&1&0&0 &0&1&0&1&0&1&0\\
		\text{Fr } 06 &0&1&0&1 &1&1&1&1&1&1&0\\
		\text{Sa } 07 &0&1&1&0 &1&1&1&1&1&1&0\\
		\text{So } 08 &0&1&1&1 &1&1&1&1&1&1&1\\
		\text{Mo } 09 &1&0&0&0 &1&0&1&0&1&0&0\\
		\text{Di } 10 &1&0&0&1 &1&1&1&1&1&1&0\\
		\text{Mi } 11 &1&0&1&0 &1&0&1&1&1&1&0\\
		\text{Do } 12 &1&0&1&1 &1&0&1&0&1&0&0\\
		\text{Fr } 13 &1&1&0&0 &1&0&1&0&1&0&0\\
		\text{Sa } 14 &1&1&0&1 &1&1&1&1&1&1&0\\
		\text{So } 15 &1&1&1&0 &1&0&1&0&1&0&1\\
		\text{Mo } 16 &1&1&1&1 &1&1&1&0&1&0&0\\
	\end{array}
\end{align*}

\paragraph{(b) Kanonische Normalformen}\mbox{}\\
\begin{align*}
	f_{1,DKNF} = & (\bar{x}_3\bar{x}_2\bar{x}_1\bar{x}_0) + (\bar{x}_3\bar{x}_2x_1x_0) + (\bar{x}_3x_2\bar{x}_1x_0) + (\bar{x}_3x_2x_1\bar{x}_0) + (\bar{x}_3x_2x_1x_0) + (x_3\bar{x}_2\bar{x}_1\bar{x}_0) + (x_3\bar{x}_2\bar{x}_1x_0)\\
	&+ (x_3\bar{x}_2x_1\bar{x}_0) + (x_3\bar{x}_2x_1x_0) + (x_3x_2\bar{x}_1\bar{x}_0) + (x_3x_2\bar{x}_1x_0) + (x_3x_2x_1\bar{x}_0) + (x_3x_2x_1x_0)
\end{align*}
\begin{align*}
	f_{2,KKNF} = & (\bar{x}_3 + x_2 + x_1 + x_0)\cdot(\bar{x}_3 + x_2 + \bar{x}_1 + x_0)\cdot(\bar{x}_3 + x_2 + \bar{x}_1 + \bar{x}_0)\\
	&\cdot(\bar{x}_3 + \bar{x}_2 + x_1 + x_0)\cdot(\bar{x}_3 + \bar{x}_2 + \bar{x}_1 + x_0)
\end{align*}

\paragraph{(c) Algebraische Minimierung}\mbox{}\\
XXX

\paragraph{(d) Karnaugh-Veitch}\mbox{}\\
Segment 3: $f_{3,KV,DNF} = x_3 + x_2x_0 + x_2x_1 + \bar{x}_2\bar{x}_1\bar{x}_0$
\begin{align*}
	\begin{array}{cccccc}
		&\bar{x}_0&x_0&x_0&\bar{x}_0&\\
		\bar{x}_1&1&0&1&0&\bar{x}_3\\
		x_1&0&0&1&1&\bar{x}_3\\
		x_1&1&1&1&1&x_3\\
		\bar{x}_1&1&1&1&1&x_3\\
		&\bar{x}_2&\bar{x}_2&x_2&x_2&\\
	\end{array}
\end{align*}
Segment 4: $f_{4,KV,KNF} = (\bar{x}_0+\bar{x}_1+\bar{x}_3)\cdot(\bar{x}_3+\bar{x}_2+x_0)\cdot(\bar{x}_3+x_1+x_0)$
\begin{align*}
	\begin{array}{cccccc}
		&\bar{x}_0&x_0&x_0&\bar{x}_0&\\
		\bar{x}_1&1&1&1&1&\bar{x}_3\\
		x_1&1&1&1&1&\bar{x}_3\\
		x_1&1&0&0&0&x_3\\
		\bar{x}_1&0&1&1&0&x_3\\
		&\bar{x}_2&\bar{x}_2&x_2&x_2&\\
	\end{array}
\end{align*}

\paragraph{(e) Quine McCluskey}\mbox{}\\
XXX

\paragraph{(f) Weniger ist mehr}\mbox{}\\
Segment 1(b): $13\cdot\text{AND}_4 + \text{OR}_7 + \text{OR}_7 + 23\cdot\text{NOT}_1 \rightarrow 208\text{Transistoren}$\\
Segment 1(c): $6\cdot\text{AND}_2 + \text{AND}_3 + \text{OR}_7 + 3\cdot\text{NOT}_1 \rightarrow 66\text{Transistoren}$\\
$\rightarrow$ Ersparnis um $(1-\frac{66}{208}\approx39.8\%)$ durch Minimierung\\
Segment 2(b): $\text{AND}_5 + 5\cdot\text{OR}_4 + 11\cdot\text{NOT}_1 \rightarrow 84\text{Transistoren}$\\
Segment 2(c): $\text{AND}_2 + \text{OR}_2 + \text{OR}_3 + 2\cdot\text{NOT}_1 \rightarrow 24\text{Transistoren}$\\
$\rightarrow$ Ersparnis um $(1-\frac{24}{84}\approx71.4\%)$ durch Minimierung\\



\end{document}
