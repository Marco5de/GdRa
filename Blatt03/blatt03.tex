\documentclass[a4paper]{article}
\usepackage[utf8]{inputenc}
\usepackage{amssymb}
\usepackage[ngerman]{babel}
\usepackage{hyperref}
\usepackage{enumitem}
\usepackage{listings}
\usepackage{amsmath}
\usepackage{esvect}
\usepackage{float}
\usepackage{graphicx}
\usepackage[table]{xcolor}% http://ctan.org/pkg/xcolor
\usepackage{todonotes}
\usepackage{pgfplots}
\usepackage{verbatim}
\usepackage{multirow}
\usepackage{booktabs}
\pgfplotsset{compat=1.10}
\usepgfplotslibrary{fillbetween}
\usetikzlibrary{patterns}
\usepackage{mathtools}
\usepackage{centernot}
\usepackage{mathabx}

\newcommand{\uproman}[1]{\uppercase\expandafter{\romannumeral#1}}
\newcommand\mathbfont{\usefont{U}{mathb}{m}{n}}

\hypersetup{
     colorlinks   = true,
     citecolor    = gray
}


\title{Grundlagen der Rechnerarchitektur Blatt 3}
\author{Marco Deuscher \and Carolin Schindler}
\date{11. November 2019}

\begin{document}

\maketitle
\section{Aufgabe: Dezimalzahlen umrechnen}
\paragraph{(a)}
\begin{align*} 
	1944_{10} & = 1024_{10} + 512_{10} + 256_{10} + 128_{10} + 16_{10} + 8_{10}\\ 
			  & = (2^{10}+2^9+2^8+2^7+2^4+2^3)_{10} = 111\;1001\;1000_2
\end{align*}

\paragraph{(b)}
\begin{align*}
	1476_{10} : 8 &= 184 \quad \text{R}=4\\
	184_{10} : 8 &= 23 \quad \text{R}=0\\
	23_{10} : 8 &= 2 \quad \text{R}=7\\
	2_{10} : 8 &= 0 \quad \text{R}=2\\
	\rightarrow 1476_{10} &= 2704_{8}
\end{align*}


\paragraph{(c)}
\begin{align*}
	1535_{10} : 16 &= 95 \quad \text{R}=15\\
	95_{10} : 16 &= 5 \quad \text{R}=15\\
	5_{10} : 16 &= 0 \quad \text{R}=5\\
	\rightarrow 1535_{10} &= 5FF_{16} 
\end{align*}

\paragraph{(d)}
\begin{align*}
	116_{10} : 7 &= 16 \quad \text{R}=4\\
	16_{10} : 7 &= 2 \quad \text{R}=2\\
	2_{10} : 7 &= 0 \quad \text{R}=2\\
	\rightarrow 116_{10} &= 224_7
\end{align*}


\section{Aufgabe: Ins Dezimalsystem umrechnen}
\paragraph{(a)}
$ 1100\;0111_2 = (2^0 + 2^1 + 2^2 + 2^6 + 2^7)_{10} = 199_{10} $
\paragraph{(b)}
$1065_7 = (5\cdot7^0+6\cdot7^1+7^3)_{10}=390_{10}$
\paragraph{(c)}
$2EA_{16} = (10\cdot16^0+15\cdot16^1+2\cdot16^2)_{10} = 762_{10}$
\paragraph{(d)}
$3262_8 = (2\cdot8^0+6\cdot8^1+2\cdot8^2+3\cdot8^3)_{10} = 1714_{10} $


\section{Zwischen Systemen umrechnen}
\paragraph{(a)}
$227_{16} = 0010\;0010\;0111_2 = 1047_8  $
\paragraph{(b)}
$10010001101_2 = 2215_8$
\paragraph{(c)}
$101011111111111101101000000001111_2 = AFFED00F_{16}$
\paragraph{(d)}
$5742_9=012\;021\;011\;012_3$


\section{Komisches Zahlensystem}
Verwende aufsteigend die folgenden Werte\\
$\{0,1,2,3,4,5,6,7,8,9,a,b,c,d,e,f,g,h,i,j,k\} $
\paragraph{(a)}
$26_{10} = (1\cdot21^1+5\cdot21^0)_{10} = 15_{21}$
\paragraph{(b)}
$19_{10} = (19*21^0)_{10} = k_{21}$


\section{Most significant bit}
\paragraph{(a)}
\begin{align*}
	1050_8 = (5\cdot8^1 + 1\cdot8^3)_{10} = 552_{10} \quad &\text{MSB left}\\
	1050_8 = (1\cdot8^0 + 5\cdot8^2)_{10} = 321_{10} \quad &\text{MSB right}\\
\end{align*}

\paragraph{(b)}
\begin{align*}
	10110010010_2 = (2^1+2^4+2^7+2^8+2^{10})_{10} = 1426_{10} \quad &\text{MSB left}\\
	10110010010_2 = (2^0+2^2+2^3+2^6+2^9)_{10} = 589_{10} \quad &\text{MSB right}
\end{align*}

\paragraph{(c)}
\begin{align*}
	4242_{10} = 4242_{10} \quad &\text{MSB left}\\
	4242_{10} = 2424_{10} \quad &\text{MSB right}\\
\end{align*}

\paragraph{(d)}
$A47_{14} = (7\cdot14^0+4\cdot14^1+10\cdot14^2)_{10}=2023_{10}$
\begin{align*}
	2023_{10} : 4 &= 505 \quad \text{R}=3\\
	505_{10} : 4 &= 126 \quad \text{R}=1\\
	126_{10} : 4 &= 31 \quad \text{R}=2\\
	31_{10} : 4 &= 7 \quad \text{R}=3\\
	7_{10} : 4 &=1 \quad \text{R}=3\\
	1_{10} : 4 &=0 \quad \text{R}=1\\
	\rightarrow A47_{10} &= 133213_4 \quad \text{mit MSB links}\\
\end{align*}

$A47_{14} = (10\cdot14^0+4\cdot14^1+7\cdot14^2)_{10} = 1438_{10}$
\begin{align*}
	1438_{10} :4 &= 359 \quad \text{R} = 2\\
	359_{10} : 4 &= 89 \quad \text{R} = 3\\
	89_{10} : 4 &= 22 \quad \text{R} = 1\\
	22_{10} : 4 &= 5 \quad \text{R} = 2\\
	5_{10} : 4 &= 1 \quad \text{R} = 1\\
	1_{10} : 4 &= 1 \quad \text{R} = 1\\
	\rightarrow A47_{14} &= 112132_4 \quad \text{mit MSB rechts}\\
\end{align*}


\section{Knobelaufgabe}
$65243_b = 27299_{10}$ mit $b<10$, da $65243_{10} > 27299_{10}$\\
Für $b=8$ erhält man dann\\
$(3+4\cdot8^1+2\cdot8^2+5\cdot8^3+6\cdot8^4)_{10} = 27299_{10}$


\section{Festkomma}
\paragraph{(a)}
$10,625_{10} = 1010,101_2$, da
$10_{10} = 1010_2$ und $0.625_{10} = (0,5+0,125)_{10} = (2^{-1}+2^{-3})_{10} = 0,101_2$

\paragraph{(b)}
$101101,1101_2 = (2^5+2^3+2^2+2^0+2^{-1}+2^{-2}+2^{-4})_{10} = 45,8125_{10}$

\end{document}
